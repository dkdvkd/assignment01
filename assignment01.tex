\documentclass[15pt]{article}
\usepackage{kotex}
\usepackage[pdftex]{graphicx}

\begin{document}
{\Large \hspace{2cm} How to use the utility git.}\newline
{\hspace*{8.3cm}융합공학부 디지털이미징}\newline
{\hspace*{9.6cm}20142740 박세인}
\newline\newline
-\textbf{분산제어 방식의 소스 제어 제품}:  \newline
중앙 서버인 저장소에 의존하지 않고, 네트워크 연결이 되지 않아도 사용 가능하기 때문에, 오프라인이나 원격으로 작업 할 수 있으며, 저장소에 문제가 생겨도 복구가 가증하다. 코드를 지역적으로 커밋고, push를 통해 동기화 된다. 서버의 복사본과 동기화를 통해 지역 사본 또한 완전한 버전 컨트롤 저장소가 된다. 
\newline\newline -\textbf{Version control system}: \newline
코드를 수정할 때, 파일의 스냅샷을 영구적으로 저장하여 변경사항들을 추적할 수 있고, 필요할 때 다시 불러낼 수 있는 것을 가능하게 하는 소프트웨어로, 모든 버전들을 아카이빙하면서도 원하는 버전만을 보여주는 것이 가능하다. 또한 이렇게 관리한 내용을 오픈소스로 공유 가능하기 때문에, 개발자의 혼란을 방지하고, 다른 사람과의 협업을 통해 진행하는 프로젝트 진행 시 용이하게 한다.
\newline\newline
-\textbf{git의 단계}:\newline
\textbf{1. patch 생성}\newline
패치는 작업 내용을 다른 branch나 다른 repository에 반영할 수 있는 형태로 만들기 때문에 충돌같은 문제만 생기지 않는다면 반영이 가능하다. commit하지 않은 내용을 패치로 만들수도 있고, commit한 내용을 피치로 만들 수도 있으며, 여러개의 commit을 하나의 패치로도 만들 수 있다. 주 작업 중인 내용을 임시로 저장하는 용도로 사용한다.\newline\newline
\textbf{2. reset}\newline
git의 현재 상태를 특정 시점으로 되돌리는 기능이다. 되돌아가고 싶은 커밋을 선택할 수 있으며, soft, mixed, hard 세 종류가 있다. hard reset은 모든 변경사항을 깔끔하게 되돌려준다.\newline\newline
\textbf{3.pull}\newline
앞에서 본인이 작업하기 이전 상태로 돌린 후, pull을 통해 다른 사람들의 최신 작업 내용들을 remote repository로부터 다운받아 local repository의 내용을 최신으로 업데이트 할 수 있다.\newline\newline
\textbf{4. patch 적용}\newline
1단계에서 생성한 패치를 반영하는 단계로, 일종의 resotre이다.\newline\newline
\textbf{5. conflict 처리}\newline
같은 commit에서 독립적이지 않은 부분을 서로 다르게 수정할 경우 생기는 현상이다. 충돌이 발생하면 git에서 자동으로 충돌이 발생한 부분을 표시해 주며, 에디터나 각종 merge tool을 사용해서 적절하게 수정을 진행해야 한다.\newline\newline
\textbf{6. commit}\newline
변경이력을 git에 저장하는 것으로, 변경사항을 그 때 그 때 저장하고, 다시 불러올 수 있다. 
\newline\newline
\textbf{7. push}\newline
결론적으로, 변경점 백업-문제없는 버전으로 리셋-최신버전으로 업데이트-변경점 복원을 통해 문제 해결 및 원활한 개발을 진행할 수 있다.\newline
\begin{figure}
\centering
\includegraphics[width=0.8\textwidth]{11.png}
\end{figure}
참고자료:\newline
https://tmondev.blog.me/220763012361 \newline
https://blog.naver.com/doblo77/220394736439\newline
\end{document}
